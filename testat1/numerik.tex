\documentclass[german]{article}
\usepackage[utf8]{inputenc}
\usepackage{german}
\usepackage{amsmath}
\usepackage{amssymb}
\usepackage{graphicx}
\usepackage{xcolor}
\setlength{\parindent}{0cm}
\setlength{\parindent}{0cm}
\author{Till Rohrmann, Johannes Reifferscheid}
\title{
	Matlab-Tutorial 1
}

\begin{document}
	\maketitle
 
    \section{Aufgabe 1b}
    
      \includegraphics[width=90mm]{qr-iteration/qrFig1.pdf}
      
      \includegraphics[width=90mm]{qr-iteration/qrFig2.pdf}
 
    \section{Aufgabe 2b}

      \includegraphics[width=80mm]{gauss-legendre/glFig1.pdf}
      
      \includegraphics[width=80mm]{gauss-legendre/glFig2.pdf}

	\section{Aufgabe 2c}

	Da es zu jeder Nullstelle $\lambda$ eines Legendre-Polynoms auch eine Nullstelle $-\lambda$ gibt, und die Nullstellen als Eigenwerte einer Matrix berechnet werden, kann aufgrund der Symmetrie der Eigenwerte der Rayleigh-Shift nicht verwendet werden (wie in der Übung gezeigt).

	\section{Aufgabe 3b}

      \includegraphics[width=120mm]{gauss-legendre2/gl2Fig1.pdf}
      
      \includegraphics[width=120mm]{gauss-legendre2/gl2Fig2.pdf}

	\section{Aufgabe 3c}

	Die Wurzelfunktion ist an der Stelle 0 nicht stetig differenzierbar, ist also keine analytische Funktion im Intervall $[0;1]$, so dass sich auch bei der Gauß-Quadratur kein exponentieller Fehlerabfall bezüglich m einstellt. Im Gegensatz dazu ist die Funktion $\frac{1}{x^2+1}$ beliebig oft stetig differenzierbar und somit analytisch, so dass nach der Fehlerabschätzung $err < C (b-a)^{m+2} \|f^{(m+1)}\|_{\infty,[a,b]}$ der Fehler exponentiell mit steigendem m abfällt.
  
\end{document}
