\documentclass[german]{article}
\usepackage[utf8]{inputenc}
\usepackage{german}
\usepackage{amsmath}
\usepackage{amssymb}
\usepackage{graphicx}
\usepackage{epstopdf}
\usepackage{xcolor}
\usepackage{float}
\setlength{\parindent}{0cm}
\setlength{\parindent}{0cm}
\author{Johannes Reifferscheid, Till Rohrmann}
\title{
	Matlab-Tutorial 2
}

\begin{document}
	\maketitle
 
   \section*{Aufgabe 1 b)}
   	\begin{figure}[H]
   		\centerline{\includegraphics[height=7cm]{cg/cgFg.eps}}
   		\caption{Anzahl der Iterationen zur approximativen Lösung einer Poissonmatrix}
   	\end{figure}
   	
   	\subsection*{Interpretation}
   		Das vorkonditionierte konjugierte Gradientenverfahren konvergiert am schnellsten mit der Matrix $N_{SSOR}$ als Vorkonditionierer.
   		Etwas langsamer, jedoch schnneller als das nicht vorkonditionierte konjugierte Gradientenverfahren,
   		konvergiert das Verfahren mit der Matrix $N_{SGS}$ als Vorkonditionierer. Im Gegensatz zu den Matrizen
   		$N_{SSOR}$ und $N_{SGS}$ zeigt das Verfahren mit dem Vorkonditionierer $N_{Jac}$ keinerlei Verbesserung
   		in der Konvergenzeigenschaft im Vergleich zum unkonditionierten Verfahren. Folglich reduziert die Matrix
   		$N_{SSOR}$ die Konditionszahl der Matrix $A$ am stärksten und es gilt in diesem Fall: 
   		$\kappa({N_{SSOR}A}) < \kappa({N_{SGS}A}) < \kappa({N_{Jac}A})=\kappa({A})$
   	
   \section*{Aufgabe 2 b)}
   
   	\begin{figure}[H]
			\centerline{\includegraphics[height=7cm]{esv/esvFg1.eps}}
			\caption{Anzahl der Beute in Abhängigkeit der Zeit}
		\end{figure}
		\begin{figure}[H]
			\centerline{\includegraphics[height=7cm]{esv/esvFg2.eps}}
			\caption{Anzahl der Räuber in Abhängigkeit der Zeit}
		\end{figure}
		\begin{figure}[H]
			\centerline{\includegraphics[height=7cm]{esv/esvFg3.eps}}
			\caption{Anzahl der Beute und Räuber}
		\end{figure}
		
	\section*{Aufgabe 2 d)}
		\begin{figure}
		[H]
			\centerline{\includegraphics[height=7cm]{esv/sbFg1.eps}}
			\caption{Steifer Stab}
		\end{figure}
		\begin{figure}
		[H]
			\centerline{\includegraphics[height=7cm]{esv/sbFg2.eps}}
			\caption{Steifer Stab}
		\end{figure}
		
		\subsection*{Interpretation}
			Das Newton-Verfahren benötigt deutlich weniger Iterationsschritte zum Finden einer approximativen Lösung
			für jeden Zeitschritt des Crank-Nicolson-Verfahrens als die Fixpunktiteration. Jedoch kann im Allgemeinen
			nicht geschlussfolgert werden, dass das Newton-Verfahren effizienter ist, weil die Berechnung eines
			einzelnen Iterationsschritt die Auswertung der Ableitung und unter Umständen das Lösen eines linearen
			Gleichungssystem beinhaltet. Darüber hinaus ist die Kenntnis der Ableitung notwendig.
			
			Auffällig ist des Weiteren, dass die Fixpunktiteration bei der Wahl einer zu großen zeitlichen Schrittweite $h$
			nicht gegen den Fixpunkt konvergiert. Folglich ist die approximative Lösung für die Differentialgleichung nicht zu gebrauchen.
			Das Newton-Verfahren zeigt diese Schwächen auch bei großer Schrittweite $h$ nicht und liefert ordentliche Ergebnisse.
\end{document}
