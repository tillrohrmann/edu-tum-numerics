\section{Integration gewöhnlicher Differentialgleichungen}

Die Behandlung von Differentialgleichungen spielt in der Numerik eine zentrale Rolle. Die mathematische Modellierung von zeitabhängigen Vorgängen führt vielfach auf Systeme von Differentialgleichungen (Biologie, Mechanik, Wirtschaft). Hängen die Zustandsgrößen nur von der Zeit ab, so spricht man von \emph{gewöhnlichen Differentialgleichungen}, hängen sie auch vom Ort ab, so spricht man von \emph{partiellen Differentialgleichungen}. Da die analytische Lösung meist nicht explizit berechenbar ist, ist man auf numerische Approximationen angewiesen.

\subsection{Aufgabenstellung und einfache Einschrittverfahren}

\begin{example} 
	Idealisiertes Pendel
	\begin{align}
		\nonumber m l \phi &= -m g sin(\phi)\\
		\nonumber \phi(0) &= \alpha\\
		\nonumber \dot{\phi}(0) &= 0
	\end{align}
	Wobei $l$ die Länge des Pendels angibt, $m$ die Masse des Massepunkts am Ende des Pendels und $\phi$ die Auslenkung des Pendels.
\end{example}

\begin{example} 
	Lotka-Volterra (Räuber-Beute-Modell)
	\begin{align}
		\nonumber\dot{y}_1 &= c_1y_1(1-\alpha_1y_2)\\
		\nonumber\dot{y}_2 &= c_2y_2(\alpha_2y_1-1)
	\end{align}
		
	Mit $c_1, c_2, \alpha_1, \alpha_2 > 0$ und Anfangswerten $y_1(0) = y^0_1$ und $y_2(0) = y^0_2$.
\end{example}

\begin{example} 
	Dreikörperproblem (Euler 1772)
\end{example}

\begin{definition}[Definition IV.1]
	Gesucht sei 
	$y(t), t \in [t_0, T], y \in \mathbb{R}^n$ mit 
	$$y^{(m)} = f(t, \dot{y}, \ddot{y}, ..., y^{(m-1)})$$
	und
	$$y(t_0) = t_0, \dot{y}(t_0) = z_1, ..., y^{(m-1})(t_0) = t_{m-1}.$$
	
	Dies definiert ein Anfangswertproblem m-ter Ordnung.
\end{definition}

Die meisten numerischen Verfahren betrachten Anfangswertprobleme 1. Ordnung. Dies ist ausreichend da jedes System der Größe n von m-ter Ordnung in ein System der Größe $n \cdot m$ von 1-ter Ordnung überführt werden kann.

$$y_1(t) = y(t)$$
$$y_2(t) = \dot{y}(t) = \dot{y}_1(t)$$
$$y_3(t) = \ddot{y}(t) = \dot{y}_2(t)$$
$$\vdots$$
$$y_m(t) = y^{(m-1)}(t) = \dot{y}_{m-1}(t)$$

Im Folgenden gehen wir von korrekt gestellten Problemen aus (Hadamard):

\begin{itemize}
	\item Existenz einer Lösung (Satz von Peano)
	\item Eindeutigkeit der Lösung (Satz von Picard-Lindelöf)
	\item Stetige Abhängigkeit der Lösung von den Daten (Lemma von Gronwall)
\end{itemize}

Von nun an betrachten wir Anfangswertprobleme 1. Ordnung der Form:

\begin{quote}
	Gesucht: 
	$$y: [t_0, T] \rightarrow \mathbb{R}^n$$ mit 
	$y = f(t, y)$, $y(t_0) = y_0$ und f Lipschitzstetig in y
	$$\|f(t,y_1) - f(t,y_2)\| \le L \|y_1 - y_2\|$$
\end{quote}

Die Grundidee aller numerischer Verfahren ist die Zerlegung in eine endliche Anzahl von Teilintervallen $[t_1, t_{i+1}]$, $i = 0, ... N-1$ mit $t_0 < t_1 < ... < t_N = T$. Mit $h_i = t_{i+1} - t_i$ bezeichnet man die Zeitschrittweite. Nun wird für jedes $i$ eine Approximation $y_i$ von $y(t_i)$ gesucht. Eine Vielzahl von Verfahrensklassen basiert auf Integration, Numerischer Quadratur, Extrapolation.

\begin{align}
	\nonumber y(t_{i+1}) - y(t_i) &= \int_{t_i}^{t_{i+1}}\!\dot{y}(t)\,dt \\
	\nonumber &= \int_{t_i}^{t_{i+1}}\!f(t, y(t))\,dt \\
	\nonumber &=: I_i
\end{align}

Man kann $I_i$ ersetzen durch

\begin{enumerate}
	\item[1)] $I_i \approx h_i f(t_i, y(t_i))$ 
	\item[2)] $I_i \approx h_i f(t_{i+1}, y(t_{i+1}))$ 
	\item[3)] $I_i \approx \frac{h_i}{2} (f(t_i, y(t_i)) + f(t_{i+1}, y(t_{i+1})))$ (Trapezregel)
	\item[4)] $I_i \approx h_i f(t_{i+\frac{1}{2}}, y(t_{i+\frac{1}{2}}))$
	\item[5)] $I_i \approx \frac{h_i}{2} (f(t_i, y(t_i)) + f(t_i, y(t_i)) + h_i \frac {f(t_i,y(t_i))-f(t_{i-1}, y(t_{i-1}))}{h_{i-1}})$
\end{enumerate}

Im nächsten Schritt werden die unbekannten Funktionswerte $y(t_l),  = i, i+1, i-1$ durch ihre jeweiligen numerischen Approxomationen $y_l$ ersetzt. Dies definiert $y_{i+1}$

\begin{enumerate}
	\item[1)] $y_{i+1} = y_i + h_i f(t_i, y_i)$ (expliziter Euler, forward Euler)
	\item[2)] $y_{i+1} = y_i + h_i f(t_{i+1}, y_{i+1})$ (impliziter Euler, backward Euler)
	\item[3)] $y_{i+1} = y_i + \frac{h_i}{2} (f(t_i, y_i) + f(t_{i+1}, y_{i+1}))$ (Crank-Nicolson)
	\item[4)] $y_{i+1} = y_i + h_i f(t_{i+\frac{1}{2}}, y_{i+\frac{1}{2}})$ (Mittelpunktsregel) \\
		Spezialfall $h_i = h$: $y_{i+\frac{1}{2}} = y_{i-\frac{1}{2}} + h f(t_i, y_i)$
	\item[5)] Sei $h_i = h$. $y_{i+1} = y_i + h (\frac{3}{2} f(t_i, y_i) - \frac{1}{2} f(t_{i-1}, y_{i-1}))$ (Adams-Bashforth Typ)
\end{enumerate}

Ersetzt man im Verfahren 3 bzw. 4 auf der rechten Seite $y_{i+1}$ bzw $y_{i+\frac{1}{2}}$ durch die Approximation aus einem expliziten Euler Schritt mit $h_i$ bzw. $\frac{h_i}{2}$, so erhält man

\begin{enumerate}
	\item[$3_{mod}$)] $y_{i+1} = y_i + \frac{h_i}{2} (f(t_i, y_i) + f(t_{i+1}, y_i + h_i f(t_i, y_i)))$ (Heun)
	\item[$4_{mod}$)] $y_{i+1} = y_i + h_i f(t_{i+\frac{1}{2}}, y_i + \frac{h_i}{2} f(t_i, y_i))$ (modifiziertes Eulerverfahren)
\end{enumerate}

Frage: 
\begin{itemize}
	\item Wie können die Methoden klassifiziert werden?
	\item Liegen Abschätzungen der Form $\|y(t_i) - y_i\| = O(h^p), h = \!\displaystyle\max_{i=0..N-1}{h_i}$ vor?
\end{itemize} 

\begin{definition}[Definition IV.2]
	Klassifizierung
	\begin{enumerate}
		\item Ein Verfahren heißt explizit, falls $y_{i+1}$ durch Funktionsauswertungen von $f$ direkt bestimmt werden kann (Verfahren 1, 4, 5, $3_{mod}$, $4_{mod}$. Andernfalls heißt es implizit.
		\item Ein Verfahren heißt Einschrittverfahren falls $y_{i+1}$ nur von $y_i$ abhängt (Verfahren 1, 2, 3, $3_{mod}$, $4_{mod}$). Hängt $y_{i+1}$ von den Werten $y_i, ..., y_{i-k+1}$ ab, so heißt es (k-Schritt) Mehrschrittverfahren.  
	\end{enumerate}
\end{definition}

\subsection{Konvergenz von Einschrittverfahren}

Ein allgemeines Einschrittverfahren kann in der Form 
$$y_{i+1} = y_i + h_i \phi (t_i, h_i, y_i, y_{i+1}; f)$$

dargestellt werden. $\phi$ nennt man dann Verfahrensfunktion oder Inkrementfunktion. 

\begin{example} Für die Verfahrensfunktion des expliziten Euler gilt:

	$$\phi_1(t_i, h_i, y_i, y_{i+1}) = f(t_i, y_i)$$
	
	Für die Verfahrensfunktion des impliziten Euler gilt:
	
	$$\phi_2(t_i, h_i, y_i, y_{i+1}) = f(t_{i+1}, y_{i+1})$$

\end{example}