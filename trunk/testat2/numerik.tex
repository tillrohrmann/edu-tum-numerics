\documentclass[german]{article}
\usepackage[utf8]{inputenc}
\usepackage{german}
\usepackage{amsmath}
\usepackage{amssymb}
\usepackage{graphicx}
\usepackage{epstopdf}
\usepackage{xcolor}
\usepackage{float}
\setlength{\parindent}{0cm}
\setlength{\parindent}{0cm}
\author{Johannes Reifferscheid, Till Rohrmann}
\title{
	Matlab-Tutorial 2
}

\begin{document}
	\maketitle
 
   \section*{Aufgabe 1 b)}
   	\begin{figure}[H]
   		\centerline{\includegraphics[height=7cm]{cg/cgFg.eps}}
   		\caption{Anzahl der Iterationen zur approximativen Lösung einer Poissonmatrix}
   	\end{figure}
   	
   	\subsection*{Interpretation der Konvergenz in Abhängigkeit der Vorkonditionierer}
   		TODO:
   		Die Kondition $\kappa(BA)$ wird am besten durch $B^{-1}=N_{SSOR}$ reduziert, weshalb die Konvergenz in diesem Fall am Besten ist.
   		
   \section*{Aufgabe 2 b)}
   
   	\begin{figure}[H]
			\centerline{\includegraphics[height=7cm]{esv/esvFg1.eps}}
			\caption{Anzahl der Beute in Abhängigkeit der Zeit}
		\end{figure}
		\begin{figure}[H]
			\centerline{\includegraphics[height=7cm]{esv/esvFg2.eps}}
			\caption{Anzahl der Räuber in Abhängigkeit der Zeit}
		\end{figure}
		\begin{figure}[H]
			\centerline{\includegraphics[height=7cm]{esv/esvFg3.eps}}
			\caption{Anzahl der Beute und Räuber}
		\end{figure}
		
	\section*{Aufgabe 2 d)}
		\begin{figure}
		[H]
			\centerline{\includegraphics[height=7cm]{esv/sbFg1.eps}}
			\caption{Steifer Stab}
		\end{figure}
		\begin{figure}
		[H]
			\centerline{\includegraphics[height=7cm]{esv/sbFg2.eps}}
			\caption{Steifer Stab}
		\end{figure}
		
		\subsection*{Interpretation}
			TODO:
			Die Fixpunktiteration scheint bei zu großer Schrittweite $h$ zu divergieren und liefert somit keine brauchbaren Ergebnisse.
			Im Gegensatz dazu liefert das Newtonverfahren auch bei großer Schrittweite $h$ gute Ergebnisse.

\end{document}
